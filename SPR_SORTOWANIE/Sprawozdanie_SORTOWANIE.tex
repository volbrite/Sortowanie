\documentclass[12pt,a4paper,oneside]{article}
\usepackage[utf8x]{inputenc}
\usepackage[T1]{fontenc}
\usepackage{amsmath}
\usepackage{amsfonts}
\usepackage{amssymb}
\usepackage{makeidx}
\usepackage{color}
\usepackage{lmodern}
\usepackage{graphicx}
\usepackage[shell]{gnuplottex}
\usepackage{pdflscape}
\usepackage{afterpage}
\usepackage{mathtools}
\usepackage[left=1.5cm,right=1.5cm,top=1.5cm,bottom=1.5cm]{geometry}
\author{Michał Bizoń}
\title{Sprawozdanie z Sortowania Liczb}
\setlength{\parskip}{2mm}
\begin{document}

\title{Sprawozdanie z Sortowania Liczb}
\author{Michał Bizoń}
\date{05 IV 2014}
\maketitle

\section{Założenia i implementacja metod}
Naszym zadaniem była implementacja algorytmów sortowania liczb - zaimplementowane zostały przeze mnie algorytmy: \textbf{QuickSort , MergeSort , IntroSort , HeapSort (wymagany do Introsorta)}, a następnie przetestowanie ich pod kątem szybkości oraz przypadków użycia - który jest najbardziej skuteczny przy danym zestawie danych.

\subsection{MergeSort - Sortowanie przez Scalanie}
Rekurencyjny algorytm sortowania danych, stosujący metodę \textbf{\emph{Dziel i zwyciężaj}}. Odkrycie algorytmu przypisuje się \textbf{\emph{Johnowi von Neumannowi}}. Czas tego sortowania wynosi \textbf{\emph{O(n*log[n])}}.

Algorytm ten:
\begin{itemize}
	\item Dzieli dane na dwie równe części
	\item Sortuje każdą część osobno
	\item Scala posortowane elementy w jeden ciąg
\end{itemize}

\subsection{QuickSort - Sortowanie Szybkie}
Jeden z popularnych algorytmów sortowania działających na zasadzie \textbf{\emph{Dziel i zwyciężaj}}. Sortowanie Szybkie zostało wynalezione w 1962 przez C.A.R. Hoare'a. Algorytm sortowania szybkiego jest wydajny: jego średnia złożoność obliczeniowa jest rzędu \textbf{\emph{O(n*log[n])}}. Ze względu na szybkość i prostotę implementacji jest powszechnie używany.

Algorytm ten:

\begin{itemize}
	\item Wybiera z tablicy element rozdzielający (tzw. \textbf{Piwot}) i przenosi elementy mniejsze od niego na lewo, a większe na prawo
	\item Po zakończonym przenoszeniu algorytm wywołuje się rekurencyjnie dla lewego podzbioru dopóki wszystkie dane w lewym podzbiorze nie zostaną posortowane. To samo dzieje się z prawym podzbiorem
\end{itemize}

Piwot jest zawsze wybierany na początku wywołania funkcji.

\subsection{IntroSort - Sortowanie Introspektywne}
Sortowanie to jest z rodziny hybrydowych, czyli do sortowania wykorzystuje inne algorytmy, które razem działają naprawdę szybko. Wykorzystuje m.in. sortowanie szybkie, sortowanie przez kopcowanie oraz sortowanie przez wstawianie. 
Na początku Algorytm dzieli tablicę na mniejsze części (wielkość ustalona odgórnie, np. 64 elementy) za pomocą procedury podziału znaną z sortowania szybkiego. Kiedy cała tablica zostanie  wstępnie posortowana, każda z nich jest sortowana poprzez kopcowanie. Całość dopełnia i zamyka sortowanie przez wstawianie, które działa bardzo dobrze dla prawie posortowanych danych.

Algorytm ten:
\begin{itemize}
	\item dzieli tablicę na mniejsze części (wielkość ustalona odgórnie, np. 64 elementy) za pomocą procedury podziału znaną z sortowania szybkiego
	\item Kiedy cała tablica zostanie wstępnie posortowana, każda z nich jest sortowana poprzez kopcowanie
	\item Całość dopełnia i zamyka sortowanie przez wstawianie, które działa bardzo dobrze dla prawie posortowanych danych.
\end{itemize}

\section{Sposób Implementacji}
Napisany przeze mnie program operuje na \textbf{2 tablicach} o wymiarach \textbf{[100][n]}, gdzie \textbf{\textcolor{red}{n}} jest aktualnie wybraną przez użytkownika wielkością. Przy każdym starcie symulacji tworzona jest \textbf{TABELA}, do której przekazywane są dane parametry:
\begin{itemize}
	\item ilość elementów w każdej tablicy (rozmiar \textbf{\textcolor{red}{n}})
	\item ilość wstępnie posortowanych elementów
	\item zakres losowanych liczb
	\item parametr „\textbf{\textcolor{red}{i}}” (od 0 do 100) – numer wiersza tablicy – wypełniając tablicę funkcją  RAND() zdarzyło się, że generował dwa wiersze o identycznych wartościach, dodanie „\textbf{\textcolor{red}{i}}” do SRAND rozwiązało ten problem
\end{itemize}
Drugą tabelą jest \textbf{TABELA\_SORTOWAN}, wypełniana początkowo wartościami \textit{NULL} o takich samych rozmiarach jak \textbf{TABELA}. Przed każdym sortowaniem następuje przypisanie wartości z \textbf{TABELA} do \textbf{TABELA\_SORTOWAN} (która jest poddawana sortowaniu, dzięki czemu przy każdym sortowaniu operujemy tym samym zestawem danych. Czasy są zapisywane do osobnej tablicy, a następnie zapisywane są do plików \textbf{*.csv} - ich nazwa jest uzależniona od wybranego algorytmu oraz opcji wybranych przed rozpoczęciem symulacji. Po każdym sortowaniu sprawdzana jest kolejność elementów – dla całej tablicy sprawdza czy element n <= n+1.

\section{Wyniki oraz wykresy}
Wykresy porównujące ogólne czasy przypisań tych algorytmów - dla zestawów danych kolejno:
\begin{center}
\textbf{[10 000 ; 50 000 ; 100 000 ; 500 000 ; 1 000 000]}\\
a także zakresu losowanych liczb - [ 0 ; 99 999 ]\\
\textcolor{red}{MergeSort} ; \textcolor{green}{QuickSort} ; \textcolor{blue}{IntroSort}\\
\end{center}
\newpage
\begin{center}
\subsection{Posortowane Odwrotnie}
\begin{tabular}{c||c|c|c}
Il. Elementów & \textcolor{red}{MergeSort} & \textcolor{green}{QuickSort} & \textcolor{blue}{IntroSort}\\\hline
10 000 & 0.217866 & 0.066815 & 0.041242\\\hline
50 000 & 1.24108 & 0.405296 & 0.232374\\\hline
100 000 & 2.65195 & 0.915541 & 0.498457\\\hline
500 000 & 15.3149 & 5.83867 & 2.87433\\\hline
1 000 000 & 31.9639 & 12.5349 & 6.25958\\
\end{tabular}
\begin{gnuplot}[terminal=eps,terminaloptions={font ",10" linewidth 3},scale=1]
set title "Zakres 99 999 | Posortowane Odwrotnie"
set datafile separator ';'
set ylabel 'Czas [s]'
set xlabel 'Ilość przypisanych danych'
set logscale x 10
set xtics ('10k' 10000, '50k' 50000, '100k' 100000, '500k' 500000, '1M' 1000000)
unset key
plot 'W/r0_wyniki_1.csv' title "MergeSort" with linespoints lt 1 , 'W/r0_wyniki_2.csv' title "QuickSort" with linespoints lt 2, 'W/r0_wyniki_3.csv' title "IntroSort" with linespoints lt 3
\end{gnuplot}

\subsection{Posortowane 25\% elementów}
\begin{tabular}{c||c|c|c}
Il. Elementów & \textcolor{red}{MergeSort} & \textcolor{green}{QuickSort} & \textcolor{blue}{IntroSort}\\\hline
10 000 & 0.279653 & 0.145696 & 0.161251\\\hline
50 000 & 1.59893 & 0.846128 & 0.905473\\\hline
100 000 & 3.40157 & 2.11195 & 2.38316\\\hline
500 000 & 19.5257 & 11.1548 & 11.6455\\\hline
1 000 000 & 40.9295 & 23.551 & 24.5284\\
\end{tabular}
\begin{gnuplot}[terminal=eps,terminaloptions={font ",10" linewidth 3},scale=1]
set title "Zakres 99 999 | Posortowane 25%"
set datafile separator ';'
set ylabel 'Czas [s]'
set xlabel 'Ilość przypisanych danych'
set logscale x 10
set xtics ('10k' 10000, '50k' 50000, '100k' 100000, '500k' 500000, '1M' 1000000)
unset key
plot 'W/r1_wyniki_1.csv' title "MergeSort" with linespoints lt 1 , 'W/r1_wyniki_2.csv' title "QuickSort" with linespoints lt 2, 'W/r1_wyniki_3.csv' title "IntroSort" with linespoints lt 3
\end{gnuplot}
\newpage
\subsection{Posortowane 50\% elementów}
\begin{tabular}{c||c|c|c}
Il. Elementów & \textcolor{red}{MergeSort} & \textcolor{green}{QuickSort} & \textcolor{blue}{IntroSort}\\\hline
10 000 & 0.254603 & 0.115298 & 0.116302\\\hline
50 000 & 1.47066 & 1.00844 & 1.06264\\\hline
100 000 & 3.10443 & 1.59201 & 1.43375\\\hline
500 000 & 17.7761 & 9.85518 & 10.1041\\\hline
1 000 000 & 37.1798 & 20.64 & 20.8247\\
\end{tabular}
\begin{gnuplot}[terminal=eps,terminaloptions={font ",10" linewidth 3},scale=1]
set title "Zakres 99 999 | Posortowane 50%"
set datafile separator ';'
set ylabel 'Czas [s]'
set xlabel 'Ilość przypisanych danych'
set logscale x 10
set xtics ('10k' 10000, '50k' 50000, '100k' 100000, '500k' 500000, '1M' 1000000)
unset key
plot 'W/r2_wyniki_1.csv' title "MergeSort" with linespoints lt 1 , 'W/r2_wyniki_2.csv' title "QuickSort" with linespoints lt 2, 'W/r2_wyniki_3.csv' title "IntroSort" with linespoints lt 3
\end{gnuplot}

\subsection{Posortowane 75\% elementów}
\begin{tabular}{c||c|c|c}
Il. Elementów & \textcolor{red}{MergeSort} & \textcolor{green}{QuickSort} & \textcolor{blue}{IntroSort}\\\hline
10 000 & 0.231321 & 0.086716 & 0.073935\\\hline
50 000 & 1.32414 & 0.772884 & 0.794843\\\hline
100 000 & 2.82 & 1.64857 & 1.71199\\\hline
500 000 & 16.1368 & 8.72012 & 8.53956\\\hline
1 000 000 & 33.9285 & 16.2207 & 12.4165\\
\end{tabular}
\begin{gnuplot}[terminal=eps,terminaloptions={font ",10" linewidth 3},scale=1]
set title "Zakres 99 999 | Posortowane 75%"
set datafile separator ';'
set ylabel 'Czas [s]'
set xlabel 'Ilość przypisanych danych'
set logscale x 10
set xtics ('10k' 10000, '50k' 50000, '100k' 100000, '500k' 500000, '1M' 1000000)
unset key
plot 'W/r3_wyniki_1.csv' title "MergeSort" with linespoints lt 1 , 'W/r3_wyniki_2.csv' title "QuickSort" with linespoints lt 2, 'W/r3_wyniki_3.csv' title "IntroSort" with linespoints lt 3
\end{gnuplot}
\newpage
\subsection{Posortowane 95\% elementów}
\begin{tabular}{c||c|c|c}
Il. Elementów & \textcolor{red}{MergeSort} & \textcolor{green}{QuickSort} & \textcolor{blue}{IntroSort}\\\hline
10 000 & 0.214803 & 0.066131 & 0.042678\\\hline
50 000 & 1.22505 & 0.52572 & 0.246116\\\hline
100 000 & 2.60647 & 1.14297 & 0.535877\\\hline
500 000 & 14.9096 & 6.69184 & 4.09729\\\hline
1 000 000 & 14.9096 & 14.1687 & 8.98775\\
\end{tabular}
\begin{gnuplot}[terminal=eps,terminaloptions={font ",10" linewidth 3},scale=1]
set title "Zakres 99 999 | Posortowane 95%"
set datafile separator ';'
set ylabel 'Czas [s]'
set xlabel 'Ilość przypisanych danych'
set logscale x 10
set xtics ('10k' 10000, '50k' 50000, '100k' 100000, '500k' 500000, '1M' 1000000)
unset key
plot 'W/r4_wyniki_1.csv' title "MergeSort" with linespoints lt 1 , 'W/r4_wyniki_2.csv' title "QuickSort" with linespoints lt 2, 'W/r4_wyniki_3.csv' title "IntroSort" with linespoints lt 3
\end{gnuplot}
\subsection{Posortowane 99\% elementów}
\begin{tabular}{c||c|c|c}
Il. Elementów & \textcolor{red}{MergeSort} & \textcolor{green}{QuickSort} & \textcolor{blue}{IntroSort}\\\hline
10 000 & 0.211942 & 0.062384 & 0.037232\\\hline
50 000 & 1.20931 & 0.500322 & 0.215102\\\hline
100 000 & 2.55391 & 1.08974 & 0.472521\\\hline
500 000 & 14.6977 & 6.75924 & 4.8308\\\hline
1 000 000 & 30.7118 & 16.355 & 9.62904\\
\end{tabular}
\begin{gnuplot}[terminal=eps,terminaloptions={font ",10" linewidth 3},scale=1]
set title "Zakres 99 999 | Posortowane 99%"
set datafile separator ';'
set ylabel 'Czas [s]'
set xlabel 'Ilość przypisanych danych'
set logscale x 10
set xtics ('10k' 10000, '50k' 50000, '100k' 100000, '500k' 500000, '1M' 1000000)
unset key
plot 'W/r5_wyniki_1.csv' title "MergeSort" with linespoints lt 1 , 'W/r5_wyniki_2.csv' title "QuickSort" with linespoints lt 2, 'W/r5_wyniki_3.csv' title "IntroSort" with linespoints lt 3
\end{gnuplot}
\newpage
\subsection{Posortowane 99,7\% elementów}
\begin{tabular}{c||c|c|c}
Il. Elementów & \textcolor{red}{MergeSort} & \textcolor{green}{QuickSort} & \textcolor{blue}{IntroSort}\\\hline
10 000 & 0.212263 & 0.062026 & 0.035904\\\hline
50 000 & 1.20672 & 0.496701 & 0.210869\\\hline
100 000 & 2.54969 & 1.08056 & 0.458917\\\hline
500 000 & 14.6631 & 6.39672 & 3.90337\\\hline
1 000 000 & 30.8038 & 15.7354 & 8.53766\\
\end{tabular}
\begin{gnuplot}[terminal=eps,terminaloptions={font ",10" linewidth 3},scale=1]
set title "Zakres 99 999 | Posortowane 99,7%"
set datafile separator ';'
set ylabel 'Czas [s]'
set xlabel 'Ilość przypisanych danych'
set logscale x 10
set xtics ('10k' 10000, '50k' 50000, '100k' 100000, '500k' 500000, '1M' 1000000)
unset key
plot 'W/r6_wyniki_1.csv' title "MergeSort" with linespoints lt 1 , 'W/r6_wyniki_2.csv' title "QuickSort" with linespoints lt 2, 'W/r6_wyniki_3.csv' title "IntroSort" with linespoints lt 3
\end{gnuplot}
\end{center}

\section{Wnioski}
\begin{enumerate}
\item Sortowania QuickSort oraz IntroSort działają prawie dwukrotnie szybciej niż MergeSort
\item Decydując się na wybór QuickSorta nie opłaca się wybierać \textbf{Piwotu} (El. Odniesienia) funkcją napisaną dla introsorta (NAJLEPSZY\_PIWOT), ponieważ w każdym przypadku czas operacji sortowania się wydłuża (nieznacznie)
\item Za uniwersalny można przyjąć algorytm Quicksort, działa bardzo szybko dla każdego rodzaju i zakresu danych
\item Z moich doświadczeń wynikło iż zakres losowań nie wpływa znacząco na szybkość algorytmów
\item Wyniki zaprezentowane w sprawozdaniu zostały przeprowadzne na Systemie operacyjnym \\\textbf{UBUNTU}, program został zbudowany przy pomocy kompilatora\textbf{ GCC}, wykorzystując 1 rdzeń procesora
\item Testy zostały również przeprowadzone w systemie \textbf{Windows 7 x64} - kompilowany za pomocą \textbf{Visual C++ 2010} z dodatkową komendą \textbf{/MP - MultiProcessor} - w przypadku użycia dwóch rdzeni IntroSort działa bardzo szybko - czas sortowania tablicy \textbf{1 000 000 elementów} oraz posortowaną w \textbf{99\%} wynosił niecałe 2 [s] (w przypadku \textbf{25\%} czas wynosił średnio \textbf{10,8 [s]})
\end{enumerate}
\begin{center}
\textbf{\\Testy zostały wykonane na komputerze o parametrach:}\\
Windows 7 x64 / Ubuntu 13.10, 4GB RAM, Intel Celeron(R) Dual-Core CPU T3500 @ 2.10GHz
\end{center}
\end{document}
